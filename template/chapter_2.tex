\chapter{CHAPTER TITLE}
\label{chp:CHAPTERTITLE}
INTRODUCTION TEXT TO THIS CHAPTER IN WHICH ALL SECTIONS ARE DESCRIBED ROUGHLY (1 SENTENCE EACH).

This chapter describes the ... In Section~\ref{sec:SECTIONTITLE}, examples are given of how to use tables and figures in MSc theses.

\section{SECTION TITLE}
\label{sec:SECTIONTITLE}

Every caption of a table (or figure) should start with a capital letter, and should end with a period. References to tables are given with a capital letter for table, as in ``(see Table~\ref{tab:EXAMPLETABLE})'' or ``in Table~\ref{tab:EXAMPLETABLE}, ...''.

\begin{table}[htb]
\centering
\begin{tabular}{|l|c|r|}
\hline % horizontal line
left aligned & centred & right aligned \\
\hline \hline
12           & 34      & 56            \\
\hline
\end{tabular}
\caption{Complete sentence describing the tabular data.}
\label{tab:EXAMPLETABLE}
\end{table}

References to figures are given with a capital letter for figure, as in ``(see Figure~\ref{fig:EXAMPLEFIGURE})'' or ``in Figure~\ref{fig:EXAMPLEFIGURE}, ...''.

\begin{figure}[htb]
% most GNUplot figures need to be rotated, width should be the same throughout the complete document, and no extension is needed
\includegraphics[angle=180,width=\textwidth]{pics/TUD_logo_zw}
\caption{Complete sentence describing the figure thoroughly.}
\label{fig:EXAMPLEFIGURE}
\end{figure}

