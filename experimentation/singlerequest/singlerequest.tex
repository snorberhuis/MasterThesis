\subsection{Single block creation}
In this experiment we try to create a block between two nodes.
This experiment validates the MultiChain to be able to correctly create a block between nodes in normal circumstances.
The experiment is run using gumby with all nodes running on a single computer.
Only two instances of MultiChain communities are started and between these two communities a block is created.
The logging of the both nodes is captured and recorded to verify the results of the experiment.

The output of the logging can be seen in Figure \ref{fig:singleblockexperiment}.
First node 1 sends a signature request to node 2.
This message is received and a block is persisted.
The hash of the block is displayed in the output.
The block is sent back as a signature response to node 1.
The block is saved by node 1 and has the same hash as shown in the output.
So the block between node 1 and node 2 is the same and a block was succesfully created.
The result of the experiment were also validated using the databases of both nodes.

The output also shows behaviour of MultiChain
to correctly exclude any other execution from entering mutual exclusive code.
The lines related to the mutual exclusion are prepended by "Chain Excl".
The nodes check if it is possible to enter the mutual exclusive part and correctly acquires
and releases the mutual exclusion token.

\begin{figure}
\begin{FVerbatim}[fontsize=\small]
1: Requesting Signature for candidate: 2
1: Chain Exclusion: signature request: False
1: Chain Exclusion: acquired, sending signature request.
1: Sending signature request.
2: Received signature request.
2: Chain Exclusion: process request: False
2: Chain Exclusion: acquired to process request.
2: Persisting sr: 2F7bTMxyJU7hZkvaBimT2bYm4bY=
2: Chain Exclusion: released after processing request.
2: Sending signature response.
1: Signature response received. Modified: True
1: Valid 1 signature response(s) received.
1: Persisting sr: 2F7bTMxyJU7hZkvaBimT2bYm4bY=
1: Chain exclusion: released received signature response.
\end{FVerbatim}
    \caption{Output of single block creation experiment}~\label{fig:singleblockexperiment}
\end{figure}
