\section{Denial of service attack}
The denial of service attack is a common attack seen on the internet
that disables a service by flooding it with requests of service.
This is also a potential attack on MultiChain particulairly because of a potential bottleneck.

\subsection{Mutual exclusive code}
A node can only perform a single operation at a time on its chain.
Multiple blocks cannot be created because they will both point to the same block and create a block.
This can be seen as an attack and should be punished as described in section \ref{sect:branch}.
So safeguards have been implemented to ensure that this does not occur.

This was done by implementing mutual exclusion for code that creates a block.
Only one block creation operation can be pending at any time.
This can be either requesting to create a new block with another peer and wait on response or
processing a request itself.
Because only a single operation can be handled at one time
this makes a node very vulnerable to a denial of service attack.
A bottleneck is introduced by design that cannot be scaled.

If a node is flooded by sufficient bogus requests to create a new block,
then it will become over burdend with these requests to service real requests or to send send out its own requests.
The node is denied service and cannot create meaningfull blocks.
Because the node becomes unresponsive to block creation requests other nodes will not trust him to sign future blocks
and will not be granted upload bandwidth.
The other node does not trust that in return he will receive a boost in his reputation and will stop collaboration.
The node under attack will also be unable to transform his own collaboration into a boost in reputation
by sending his own block creation requests.

The proposed denial of service attack is more sophisticated than typical denial of service attacks.
These typicaly involve simply flooding a server with requests,
but for the proposed attack the requests have to be crafted with care.
They need to valid to be serviced by the attacked node and reach the mutual exclusive part
that is the vulnerable bottleneck.
A request has to be a counterpart of a real interaction or the request can be easily filtered.
The filtered request will still impose a computational and network burden on the node,
but this is always a vulnerability and not a specific vulnerability of MultiChain.

\subsection{Filtering requests}

Detection can be implemented that will help in detecting fake requests.
The detection can be run in parallel and does not have to enter the mutual exclusive part.
Any request that is fake will not reach the mutual exclusive part
and will not drown out the service of valid block creation.
Effective analyses have to be researched and implemented in future work to harden the system to this attack.

