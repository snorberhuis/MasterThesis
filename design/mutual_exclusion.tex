\section{Single operations on the chain}
\label{sect:deadlock}
In a chain only a single block of a peer is allowed to point to a previous block.
A peer cannot have multiple blocks belonging to him all pointing to the same previous block.
As this is a potential attack as described in section \ref{sect:branch}.
The chain of a peer can only be moved forward by a single block at any time.
Only after a new block is created is the new hash available to be used in the next block.

To ensure this happens correctly the MultiChain community contains mutual exclusive code
that excludes any new operations on the chain if an operation is already pending.
The mutual exclusion is achieved by having to acquire an atomic token to allow to perform an operation on the code.
The MultiChain community will receive incoming signature requests
or requests by other parts of Tribler to send out an outgoing signature request.
The community will decline this request if the token is not available
returns execution to other parts of Tribler.
The token can be unavailable while waiting on another peer in the network to finish responding to a signature request.

When a MultiChain peer has a pending signature request,
then the peer itself will not respond to incoming signature requests from other peers.
These peers themselves will also not respond as they have a pending signature request.
This can create a circular dependency on the availability of the token.
If two peers send a request to each other at the same time, they will wait on each other.
This could result in a deadlock.

MultiChain prevents this deadlock to occur by allowing a transaction to fail
as explained in section \ref{des:halfsigned}.
If MultiChain gets into this potential deadlock one of the peers will eventually time out of their own signature request
and process the incoming request resolving the circular dependency.
The deadlock is recovered and both peers can continue operation.

This situation has occured during experimentation and it is explained in section \ref{sect:deadlock-exp}.
It is shown that MultiChain correctly recovers the potential deadlock.

