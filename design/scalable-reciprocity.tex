\subsection{Scalable reciprocity}
\label{sect:scalable-reciprocity}
One of the main pillars of the design is to have a transaction history for every peer.
This is in contrast to not distribute a global, full transaction history containing the transactions of every peer.
For example, used in the design of the blockchain of Bitcoins, discussed in section \ref{sect:bitcoin}.
The reasoning behind the idea to abandon this type of design is that a common, full truth
will become the bottleneck in the system.
This will limit the amount of interactions that can be processed
or will limit the participiation of less powerfull machines.
Commonly, this way of design is used to prevent double spending.
We believe that it is possible and much cheaper to detect and punish double spending than prevent double spending.

The reason for the limitation is that every interactions will have to be distributed to every peer in the network.
Every transaction has to be processed by every node at the cost of bandwidth, compute power and storage.
The cost might be very limited for a single transaction,
but with greater scale these cost will add up.
The amount of these three resources is limited and will limit the amount of transactions that can be processed.
This problem can be seen to affect Bitcoins and has been demonstrated in section \ref{bitcoin-limit-size}.

Every node has its own transaction history and only needs the transaction history of its peers it interacts with.
Within MultiChain peers can try to congregrate the full transaction history or only a subse.t
The flexibility provides MultiChain with unbounded scalability.
The amount of bandwidth, compute power and storage is limited to the minimal needed amount
that is needed to process only the relevant transactions.
This will allow low-powered devices to keep participating when they only have a low volume of transactions.
It also allows higher volumes of transactions in the system as a whole.