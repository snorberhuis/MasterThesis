\section{Block size}
The size of a block is important as it determines how fast the total size of a chain will grow
and the utilization of the bandwidth of network to transfer blocks.
The size is dependend on the size of the parts of the blocks
and can be determined by the choice of cryptographic primitives.
Hashing function turn a variable message string to a fixed size message digest\cite{VanderLubbe-crypto}.
The fixed size is dependend on the choice of hashing function.
Similair cryptographic signing functions deterimine the size of a signature.
An overview of the size of the parts of a block can be seen in Table: \ref{table:block_size}.
The total size of a block is 240 bytes.

\begin{table}[]
\begin{adjustwidth}{-.5in}{-.5in}
\begin{center}
\begin{tabular}{lll||lll}
Name              & Type             & Bytes                   & Name              & Type             & Bytes \\ \hline
Uploaded MBytes   & unsigned integer & 4                       & Downloaded MBytes & unsigned integer & 4     \\
Total Up A        & unsigned integer & 8                       & Total Up B        & unsigned integer & 8     \\
Total Down A      & unsigned integer & 8                       & Total Down B      & unsigned integer & 8     \\
Prior Record A    & SHA1 digest      & 20                      & Prior Record B    & SHA1 digest      & 20    \\
Sequence Number A & signed integer   & 4                       & Sequence Number B & signed integer   & 4     \\
Public Key A      & EC binary format & 64                      & Public Key B      & EC binary format & 64    \\
Signature A       & EC signature     & 40                      & Signature B       & EC signature     & 40
\end{tabular}
\caption{Block size}
\label{table:block_size}
\end{center}
\end{adjustwidth}
\end{table}

An overview of every attribute and the size of the attribute inside the database can be seen in Table \ref{table:block_size_persistence}.
SQLite can grow the size of integers in its database to fit the required size
and therefore it can be in the range of $1,2,3,4$ bytes.
Because of this dynamic allocation and because public keys are stored in the Dispersy database
the total size is smaller.

\begin{table}[]
\begin{adjustwidth}{-.5in}{-.5in}
\begin{center}
\begin{tabular}{lll||lll}
Name              & Type             & Bytes                  & Name              & Type             & Bytes    \\ \hline
Uploaded MBytes   & unsigned integer & 1,2,4                  & Downloaded MBytes & unsigned integer & 1,2,4    \\
Total Up A        & unsigned integer & 1,2,4,8                & Total Up B        & unsigned integer & 1,2,4,8  \\
Total Down A      & unsigned integer & 1,2,4,8                & Total Down B      & unsigned integer & 1,2,4,8  \\
Prior Record A    & SHA1 digest      & 20                     & Prior Record B    & SHA1 digest      & 20       \\
Sequence Number A & signed integer   & 4                      & Sequence Number B & signed integer   & 4        \\
Mid A             & SHA1 digest      & 20                     & Mid B             & SHA1 digest      & 20       \\
Signature A       & EC signature     & 40                     & Signature B       & EC signature     & 40
\end{tabular}
\caption{Block size inside the database.}
\label{table:block_size_persistence}
\end{center}
\end{adjustwidth}
\end{table}

\subsubsection{Upper limit uploaded and downloaded MB}
The maximum integer size of the total up and total down impose an upper limit
on the total uploaded and downloaded MB that MultiChain can track.
This limit is imposed by SQLite and is $2^{62}$.
Bigger numbers cannot be saved in SQLite using a native datatype.
The upper limit for MultiChain is therefore 4.612 \ensuremath{\times 10^{9}} petabytes.
This is clearly more than sufficient.