\chapter{Problem Description}
In a distributed system nodes will have interactions with peers.
A node will have to decide how to react to these peers.
Our work is beneficial to deciding how to react to these peers.

In this chapter the problem will be described of reacting to peers.
It will be explained in a simplified form.
Using this simplification, the problem will be transformed gradually
to the real problem faced in distributed systems today.

\section{Deciding to help}
Nodes can decide to cooperate with a peer or choose to not help a peer and defect.
This is the traditional Prisoner's Dilemma 
and we will explain this dilemma\cite{Nowak-PrisonerDilemma}\cite{Lai-Incentives}.
A node will receive a certain utility based upon his decision, which is either a benefit or a cost.
Nodes can help each other at a cost, but the recipient of the help will receive a benefit from the help.
The benefit received is greater than the cost and is denoted by $R$ for reward.
But if one node choose to not help, then he will receive more benefit and at no cost, $T$ for temptation.
The node that provided the aid will now receive no benefit and only incur a cost $C$.
If both nodes choose to not help each other, 
then they will both receive a penalty $P$, which is higher then the cost of helping each other.
The utility received can be seen in Table \ref{tab:pd-um}.

\begin{table}[h]
\center
	\begin{tabular}{l|ll}
	A\textbackslash B       & cooperate  & defect     \\ \hline
	cooperate & $R_A /R_B$ & $T_A /C_B$ \\
	defect    & $T_A /C_B$ & $P_A /P_B$
	\end{tabular}
\caption{Prisoner's Dilemma utility matrix}
\label{tab:pd-um}
\end{table}

This dilemma can be repeated several times with the same nodes and is the Iterated Prisoner's Dilemma.
Each time both nodes will have to decide if it will help the other node.

A node wants to receive maximum benefit at a minimal cost.
At first it might seem that a rational node will always choose to not provide aid,
because it will never incur a cost and receive maximal benefit.
But the other node will be reluctant to help a node if the aid is never returned.
Simple strategies, like tit-for-tat or win-stay, lose-shift, suffice in this scenario
and will perform well\cite{Nowak-Cooperation}.

In a large scale, distributed system, this dilemma occurs with every interaction between nodes.
The node can already be familair with the peer,
But more often the peer will be a peer the node has not interacted with before.
A further commplication is that help is one way and can no longer be exchanged.
This complication excludes the direct opportunty to barter for help 
or to barter for help in the future\cite{Lai-Incentives}.
The performance of the tit-for-tat or win-stay, lose-shift strategies
quickly deterioriate in such a situation.

For a node it is easier to abuse the generousity of others in this more anonymous situation.
Nodes that help others will be penalized through the cost they incur
and incentivized to adopt the malicious behaviour themselves.
Nodes in general will become more reluctant to help nodes\cite{Nowak-PrisonerDilemma}.
In the end no node will help and all nodes will receive a penalty.
This is commonly called the Tragedy of the Commons in the literature \cite{hardin-tragedy}.
The whole network will actually receive more benefit in total if everyone corporates.
but nodes have no way of knowing if the peer they meet are willing to help.

\section{History of decisions}

In the Iterated Prisoner's Dilemma the history of the previous transactions can be used 
to track the previous response of a node in past transactions.
Adding a history improves the performance of even the simple strategies previously mentioned.
%TODO: #35 check
A history can be used to create a currency or a reputation system.
The node providing help will receive a boost in currency or a beneficial reputation, 
that can be used in the future to receive aid.
In a currency system, receiving help will transfer currency to the helper.
In a reputation system, only nodes with a sufficiently good reputation will be helped.

The currency or reputation has to be made publicy available to all nodes in the network.
But a node publicizing to hold a certain reputation is not trustworthy.
So a interaction history, that contains every prior interaction a node has conducted, is publicized.
Nodes can check the interaction history and calculate the amount of currency 
or the reputation that a node has.
Based upon this calculation, the node can decide to provide aid or not.

The interaction history has to be distributed among the nodes in the network
to become publicly available.
Efficient distribution protocols are a difficult challenge themselves and outside the scope of our work.
But distribution is worth mentioning, because it puts constraints on the interaction history.
The interaction history has to be distributable in an efficient manner.
Only if it meets this constraints, will an interaction history be usable in a distributed system.

\section{Tamper-proofness to facilitate thrust}

Interaction histories only prevent direct abuse of the generousity of the nodes.
A malicious node can still try to tamper with the interaction history.
Example of these attacks are double spending attacks.
If the interaction history is not resilient to these types of attacks, 
no one will trust the history as fraud can happen.

Tamper-proof interaction histories prevents classes of attacks that involve altering the interaction history.
A digital currency system builds upon trust just as much as a banknote currency.
A system cannot be fully secure, 
but still a reasonable certainty is required that no tampering of the interaction history can occur.
