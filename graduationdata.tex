%De aankondiging bevat de spreker, titel, plaats, datum en tijd, samenstelling van de afstudeercommissie en een korte samenvatting (maximaal 25 regels).
\thispagestyle{empty}

\noindent \textbf{Author}\\
\begin{tabular}{l}
Steffan D. Norberhuis\\
\\
\end{tabular}\\
\noindent \textbf{Title}\\
\begin{tabular}{l}
DoubleEntry: Tamper-proof Interaction History\\
\\
\end{tabular}\\
\noindent \textbf{MSc presentation}\\
\begin{tabular}{l}
% <MM> DD, YYYY (like \today)
%TODO: #36 GRADUATION DATE\\
\\
\end{tabular}

\vspace{1.1cm}

\noindent \textbf{Graduation Committee}\\
\begin{tabular}{ll}
%TODO: #37 GRADUATION COMMITTEE & Delft University of Technology \\
% The order of listing the names: Graduation prof, supervisor(s), others ordered by title + alphabetical
%examples:
%prof. dr. ir. H. J. Sips (chair) & Delft University of Technology \\
%ir. dr. D. H. J. Epema           & Delft University of Technology \\
Prof. dr. ir. H. J. Sips            & Delft University Of Technology \\
Prof. dr. ir. J. A. Pouwelse        & Delft University of Technology \\
\end{tabular}

\begin{abstract} %de abstract bevat alleen een korte samenvatting van de inhoud van het onderzoek
Peer-to-peer networks are often large, collaborative networks where peers can join openly.
In these networks peers help other peers often in singular interactions and without direct reciprocity.
Peers can abuse and freeride.
The network without measures can fall into a tragedy of the commons
where no one helps another and only takes advantage of the generousity of peers.
Only if the reputation of a peer is publicly available at scale and tamper proof can a peer-to-peer network
escape the problems of freeriding and attain high utility for all participants.

This thesis focuses on creating the first step for a tamper proof reputation system within Tribler.
Tribler is a peer-to-peer BitTorrent system developed at the Delft University of Technology.
This first step, made by this thesis, is to design and implement a proof-of-concept bookkeeping system MultiChain.
MultiChain tracks the upload and download amounts of peers to eliminate free riding.
This bookkeeping system has to be scalable to be able to process enough transactions.

A new design of a distributed data structure that can be used as a ledger is introduced by this thesis.
This first step with MultiChain is already more resiliant to tampering than previous work.
The design of MultiChain is to have a chain of blocks for every peer.
A block contains a transaction shared between a peer.
This makes both chains of the peers intertwined and entangled between peers.
Abandoning a global ledger leads to a scalable data structure.
The implementation of the design is tested and experimented with within this thesis to validate it to correctly work.
Furthermore, a number of weak points are discussed that are future steps in creating a tamper proof reputation system.

\end{abstract}

\clearpage

