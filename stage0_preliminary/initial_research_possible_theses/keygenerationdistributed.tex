\section{Distributed key generation in a malicious network}
This field of research was thought up by the author himself while daydreaming during the Security and Cryptography course.
This idea seem to be a hurdle in other technological challenges and seems interesting.

\subsection*{Problem summary}
In a distributed network or a smart grid nodes can share a common key to operate.
This can be potentially a partial key.
It is good practice to rotate this key to minimize the threat when the key is compromised.
But it is preferable to not depend on a central node for key generation in a master-slave setup.
This is because it introduces a single point of failure or a dependency on a company to keep supporting a product.

The preliminary step before using this key is the key generation in the distributed network.
It is possible that this key generation has to take place inside a malicious network.
\subsection*{Directions of research}
The direction of research is how to do this using distributed algorithms and key generation algorithms in a combined fashion.

\subsection*{Threats}
A threat to this thesis is that it is unclear at the time if it is a field with unsolved problems,
if it is of any interests to solve these problems, and finally how likely it is to solve these problems.
A possibility is to develop an algorithm for a different key type.

\subsection*{Related work}
There is a paper proposing an algorithm for RSA keys~\cite{boneh1997efficient} where at the end $N=pq$ is known,
but no party knows $p$ or $q$. Each party holds a share of the private exponent.
A similar protocol is proposed for RSA~\cite{gilboa1999two}.

