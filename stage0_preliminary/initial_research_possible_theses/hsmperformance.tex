\section{Performance of Hardware Security Modules for key generation and retrieval}

\subsection*{Problem summary}
Hardware security modules (HSM) are devices that can provide extra security for digital keys and perform cryptographic functions.
HSMs provides physical security, active detection of tampering and tampering evidence. 
They can actively respond to tampering with for example deletion of the keys inside the device.
They are employed typically in mission-critical, clustered environments.
Some have the ability to execute custom modules within the HSM.

This problem was proposed as potential subject by a startup company.
The company is going to cluster HSMs and use them to perform key generation and key retrieval.
But the company has no prior experience with the performance of HSMs
and they want to know what performance can be expected from the HSMs.
Particular of interest is in how the performance scales with increase of a customer base and increase of the HSM cluster.

\subsection*{Directions of research}
The first step of research is to empirical evaluate the performance of clustered HSMs 
Using the evaluation proposals of improvement should be done, developed and further evaluated.

The growth of key length is also a possible direction of research. 
At longer key sizes the keys processed per minute by a HSM will be lower,
but it is unknown how much statistically this slowdown factor is. 

\subsection*{Threats}
\begin{itemize}
\item Shifting priorities in the dynamic environment of the startup that will put less importance on completing the thesis.
\item Unavailability of very expensive hardware to conduct experiments and development on.
\item The balance between Intellectual Property, insight into pricing model of the company and the requirement of publishing the master thesis report.
\end{itemize}

\subsection*{Related work}
On the specific topic of HSM clusters is not much published.
Some manufacturers provide simple tables of performance of a single HSM.
Amazon AWS provides HSM clusters inside a private virtual cluster,
and refers to these tables as a reference of performance.

A semester project was found by a student of the ETH was found on HSM clusters that included performance testing~\cite{hsmperformance}.
This was research done conducted on a new generation of HSM being rolled out inside a large cluster.

On performance research of clusters a vast amount of knowledge can be found.
These methodologies could be applied for this research topic.
