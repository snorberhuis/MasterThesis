\begin{abstract} %de abstract bevat alleen een korte samenvatting van de inhoud van het onderzoek
Peer-to-peer networks are often large, collaborative networks where peers can join openly.
The essence of a collaborative, distributed system is that every node performs tasks for other nodes.
The peers often help in singular interactions and without direct reciprocity.
Malicious peers can abuse and freeride the public goods.
The network without countermeasures can fall into a tragedy of the commons
where no one helps another and everyone takes advantage of the generosity of peers.
Only when the reputation of a peer is publicly available at scale and peers trust this reputation
can the network escape the problems of freeriding and attain high utility for all participants.

This thesis focuses on designing and implementing the first step of a tamper proof reputation system within Tribler.
Tribler is a peer-to-peer BitTorrent system developed at the Delft University of Technology.
This first step, made by this thesis, is to create MultiChain, a proof-of-concept bookkeeping system.
MultiChain tracks the upload and download amounts of peers to eliminate freeriding.
MultiChain is cryptographically protected and validated.
The bookkeeping system has to be scalable to be publicly available and be able to process enough transactions.
The system has to work in an asynchronous network.

A new design of a distributed data structure that can be used as a ledger is introduced by this thesis.
This first step with MultiChain is already more resilient to tampering than previous work, like BarterCast.
BarterCast has no security measures against tampering records.
The design of MultiChain is to have a chain of blocks for every peer as a ledger.
Peers are participants of a peer-to-peer network.
A block contains a transaction between two peers.
This block is shared and added to both chains.
This makes both chains of the peers intertwined and entangled at a shared block.
The proposed design abandons the typical global, full ledger.
The protocol of creating these blocks between peers is described.
The problems faced by MultiChain in an asynchronous network are explained.
The thesis proposes how the design can overcome these problems
by only allowing atomic operations to be performed on the chain
and to introduce unfinished blocks in the chain.

The implementation of the design is tested and experimented with within this thesis to validate it to work correctly.
Furthermore, a number of weak points are discussed.
These weak points have to be addressed in the future to create a tamper proof reputation system.
