\section{Software engineering tests}
MultiChain is tested in several ways to verify it is correctly working
following standard software engineering practices.
The enforcement of these types of tests is a policy recently introduced within Tribler.

\subsection{Unit Tests \& Integration tests}
Tribler uses Python unit tests to validate small components of code.
The tests can be run locally and
are automatically run on a Jenkins build server~\cite{jenkins,jenkins-tribler}.
Unit tests were added to increase stability of MultiChain.
Integration tests were added to test multiple components of MultiChains working integrated together.

Tribler does prove to be hard to test using unit tests.
This is due to high coupling of code within Tribler.
But mocking of classes helped in testing difficult to test code.
The separate unit tests for the conversion, payload and database were the first of it types inside Tribler.
Also Dispersy was expanded to introduce new functionality to make it easier to create unit tests for communities in the future.
An overview of the coverage can be seen in Table \ref{tab:tests}
In the end a high level of testing has been achieved, especially in comparison to the rest of Tribler.
The total test coverage of Tribler was only 16\% at time of writing~\cite{jenkins-tribler}.
A decision using code coverage tools was made that the untested code has little value to further test in comparison to the work required.
The code was tested in Gumby scenarios.

\begin{table}
\centering
\begin{tabular}{l|ll|ll}
Filename   & LOC & \% covered   & Conditionals & \% covered    \\ \hline
Community  & 187 & 81\%         & 37           & 95\%  \\
Conversion & 60  & 100\%        & 6            & 100\%  \\
Database   & 100 & 100\%        & 6            & 84\%  \\
Payload    & 89  & 100\%        & 2            & 100\% \\ \hline
Total      & 672 &              & 47           &
\end{tabular}
\caption{Unit tests coverage of MultiChain.}
\label{tab:tests}
\end{table}

\subsection{Gumby}
\label{subsect:gumby}
Next to that, Tribler uses a homemade testing environment Gumby.
Gumby can start multiple instances of Tribler and follow test scenarios.
Gumby can be used to perform system tests and experiments.
These system tests have to be manually validated.
Several scenarios have been written to validate MultiChain.
These run MultiChain either in a standalone version or integrated into the TunnelCommunity.

One of these scenarios can be found in Figure \ref{fig:exp-gumby-scenario}
In this example basic block creation is tested.
Normal situations are tested,
but also situations where the signature requests are answered late
and other requests arrive at the requesting peer at the same time.
Additionally, signature requests are controlled to not be answered at all.
During the whole test the crawler is active and scrapes the network for unknown blocks.
The notation in the scenario is the time an action has to be taken place,
the action that has to be taken, and by who if necessary.
The "@0:" can be ignored, but is required in the Gumby format.

\begin{figure}
\begin{FVerbatim}[fontsize=\small]
@0:0 set_master_member 3081a7301006072a8648ce ... 2b51
@0:0 set_community_class MultiChainNoResponseCommunity {4}
@0:0 set_community_class MultiChainDelayCommunity {5}
@0:0 set_community_class MultiChainCommunityCrawler {6}
@0:0 set_community_class {6}
@0:0 start_dispersy
@0:1 online
@0:5 reset_dispersy_statistics
@0:10 annotate start-experiment-1-peer
@0:15 introduce_candidates
@0:80 request_signature 2 {1}
@0:84 request_signature 1 {2}
@0:94 request_signature 4 {1}
@0:95 request_signature 1 {3}
@0:104 request_signature 5 {1}
@0:106 request_block 1 5 {6}
@0:110 close
@0:111 stop_dispersy
@0:112 stop
\end{FVerbatim}
    \caption{One of the Gumby definition files.}
    \label{fig:exp-gumby-scenario}
\end{figure}