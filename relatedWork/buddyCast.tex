\section{BarterCast}
BarterCast is the system used by Tribler to keep track of reputations\cite{pouwelse-buddycast}\cite{meulpolder-bartercast}.
It is fully decentralized in contrast to previous reputation systems in a peer-to-peer network.
Private BitTorrent communities for example depend on central servers track reputations.
%TODO source
BarterCast was not designed with the aim to be fully resistant to malicious nodes
that want to falsify their reputations.
The initial version has been first deployed in June 2006
and subsequently has been improved.
BarterCast will be briefly explained and its vulnerabilities.

\subsection{Epidemic protocol}
BarterCast uses BuddyCast as a protocol to build a reputation system.
BuddyCast is an epidemic protocol stack that allows spreading information of bartering activity.

An epidemic protocol works in a very simple way relying on a receive-and-forward primitive.
Information is exchanged between peers and forwarded between peers.
This is usually called gossiping.
BarterCasts gossips on the donation of upload bandwidth of other peers
and their consumption of download bandwidth.

BuddyCasts spreads information of other peers with initial messages.
This is done very 15 seconds, but a peer is only contacted every 4 hours to avoid contacting too often.
These messages contain a prefered content list, a list of peers with similair prefered content
and a list of random peers.
For each peer a public key, IP adress, port number and last seen timestamp is provided.
The peers with a similair preference are called taste buddies.

BuddyCasts uses several techniques to improve the peer selection efficiency.
Peer selection efficiency is the percentage of successfully delivered outgoing messages to peers.
This metric measures how well BuddyCasts selects peers on availablity and connectablity.
These are important because they determine how well BuddyCasts is able to handle with peer failure and exit.

The most obvious improvement that has been made is not to forward any offline peers.
BuddyCast maintains a live overlay
and continuously verify the online status of 10 random peers and 10 taste buddies.
The random peers are updated to ensure they are in fact random peers.

Peers can detect their own connectivity issues.
While they are able to connect outbound,
no incoming connections can be excepted.
These peers can improve the peer selection efficiency
by broadcasting that they are having connectivity problems
and instructing others to not gossip their identity.

Each peer collects information on how much other peers have downloaded and uploaded.
This information is forwarded in a barter record that contains the information for 10 peers.
A peer has a chance of $p$ to be random peer or a peer with similair taste for content.
The information is attached to BuddyCast messages and forwarded.

The aggregrated information contains information about direct interactions with other peers,
but also information about interactions between other peers gossiped.
The information can be visualised using BarterBowser.
A screenshot can be seen in Figure \ref{fig:barterbowser}.
The center contains the local peer.
Every circle contains peers for a certain degree.
The first circle from the centre contains peers that the local peer has direct interactions with.
The second circle contains peers that peers from the first circle has direct interactions with and so forth.

The information can be used to calculate the ratio between amount of downloaded data and uploaded data by a peer.
This shows how well or how poorly a peer has helped a network.
The calculation is done without relying on a third party.
No authority verifies the validity of the received information.
Also there is no insurance that the information is complete.
The available information can be incomplete.

\subsection{Limitations}

\subsubsection{Collusion}

\subsubsection{Sybil Attack}


This is one of the main problem with BarterCast.




